\section{Diskussion}
\label{sec:Diskussion}
Um die Ergebnisse des Versuchs diskutieren zu können wird der Mittelwert der Debye-Temperaturen 
aus Kapitel \ref{sec:DT1} mit dem Ergebnis aus Kapitel \ref{sec:DT2} verglichen. 
Dazu wird der relativen Fehler nach der Formel
\begin{equation*}
  \tilde{x} = \frac{ \lvert x_{t} - x_{e} \rvert}{\lvert x_{t} \rvert}
  \cdot 100 \%
\label{eq:relf}
\end{equation*}
berechnet, dabei bezeichnet $x_{t}$ den theoretisch berechneten Wert aus Kapitel \ref{sec:DT2} und 
$x_{e}$ den experimentell bestimmten Wert. Der Mittelwert der experimentell bestimmten 
Debye-Temperatur beträgt \SI{318.8(1)}{\kelvin}.
Daraus folgt ein relativer Fehler von \SI{4.11}{\percent}. Die Abweichung kann dadurch erklärt 
werden, dass die Temperatur des Kupfer-Zylinders nicht immer genau dem der Probe entsprach und 
deshalb die Wärmestrahlung nicht unterdrückt wurde. Dennoch ist zu bemerken, dass die Abweichung 
relativ gering ist, wenn in Betracht gezogen wird, dass die zur Berechnung verwendeten Größen 
Fehler in der gleichen Größenordnung haben. 

