\section{Theorie}
\label{sec:Theorie}
\paragraph{Ziel}
In diesem Versuch soll durch die Messung der Molwäre einer Metallprobe (??? kupfer)die Debye-Temperatur
ermittelt werden.\\
Um die Temperatrabhängigkeit der Molwärme von Festkörpern zu erklären werden im folgenden drei Modelle
vorgestellt.
\paragraph{Die klassiche Theorie}
besagt nach dem Äquipartitionstheorem, dass jedes Atom eines Körpers im Mittel die Energie
$ \left< E \right>=\frac{1}{2}\symup{k_\symup{B}}T$ für
jeden Freiheitsgrad besizt.
Klassisch wird für einen kristallinen Festkörper erwartet, dass die mittlere Energie dabei der
 mittleren potentiellen Energie entspricht, da die Atome harmonisch um ihre festgelegte Ruhelage schwingen.
 Pro Mol des Materials ergibt sich also eine Energie von
 \begin{equation}
   E=3\symup{k_\symup{B}}\symup{N_\symup{L}}T=3\symup{R}T .
   \label{eq:EproMol}
 \end{equation}
Dabei ist in \ref{eq:EproMol} $\symup{N_\symup{L}}=2,69\cdot10^{25}\, \frac{1}{\symup{m}^3}$ \cite{Lohschmidt} die Lohschmidsche Zahl
und R=$8,31\,\frac{\symup{J}}{\symup{mol\,K}}$ \cite{Gaskonst} die allgemeine Gaskonstante.\\
Für die spezifische Molwärme folgt daraus
\begin{equation}
  C_V = \left(\frac{\partial U}{\partial T} \right)_V = 3R \, .
  \label{eq:MolKla}
\end{equation}
Obwohl sich zeigt, dass $C_V$ sich dem absoluten Wert aus Gleichung \ref{eq:MolKla} bei hohen Temperaturen annähert,
liefert dieses Modell keine Erklärung zur Temperatur- und Materialabhängigkeit. Dies widerspricht den experimentellen
Beobachtungen.
\paragraph{Das Einstein-Modell}
berücksichtigt die quantenmechanische Energiequantelung der atomaren Oszillatoren.
Es besagt, dass die Energie, der mit der einheitlichen Frequenz $\omega$ schwingenden Atome sich nur um ganzzahlige Vielfache der
Energie $\hbar\omega$ ändern kann. Um die mittlere Energie zu bestimmen wird die Boltzmann-Verteilung verwendet um die
Wahrscheinlichkeit anzugeben, dass der atomare Oszillator eine Energie von
$n \hbar \omega$ mit $n \in \symup{N_0}$ besitzt:
\begin{equation}
   W(n) = \symup{exp}\left(- \frac{n \, \hbar \, \omega}{k_B \, T} \right) \, .
  \label{eq:Boltzi}
\end{equation}
Wird über alle, mit Gleichung \ref{eq:Boltzi} gewichteten, möglichen Energien summiert und durch die Summer aller Wahrscheinlichkeiten
geteilt ergibt sich für die mittlere Energie
\begin{equation}
  \left< U \right >_\symup{Einstein} = \frac{\hbar \omega}
  {\exp{\left(\frac{\hbar \omega}{\symup{k_\symup{B}} T} \right)} - 1} .
  \label{eq:EnergieEin}
\end{equation}
Für die Molwärme folgt daraus
\begin{equation}
  {C_{V,\symup{Einstein}}} =
3\symup{R} \frac{\hbar^2 \omega^2}{\symup{k_\symup{B}}^2 T^2} \frac{\exp{\left(\hbar \omega / \symup{k_\symup{B}} T \right)}}
{\left[\exp{\left(\hbar \omega / \symup{k_\symup{B}} T \right)} - 1 \right]^2}
  \label{eq:MolEin}
\end{equation}
Es wird ersichtlich, dass diese Näherung Temperaturabhängig ist und sie für hohe Temperaturen gegen 3R geht, trotzdem
zeigt sich im Vergleich mit experimentellen Werten, dass Abweichung im Bereich tiefer Temperaturen auftreten.
\paragraph{Das Debye-Modell} liefert eine genauere Aussage über die Molwärme, da es die Verteilung der Frequenzen $Z(\omega)$ aller
Oszillatoren mitbetrachtet. Wenn Materiallien mit anisotropem und dispersivem Verhalten sowie die
Frequenz- und Richtungsabhängigkeit der Phasengeschwindigkeit vernachlässigt werden, lässt sich
$Z(\omega)$ aus der Anzahl der Eigenschwingungen im Frequenzintervall $[\omega +\symup{d}\omega]$ zu
  \begin{equation}
  Z(\omega) \symup{d} \omega = \frac{L^3}{2\pi^2} \omega^2
  \left(\frac{1}{v_\symup{l}^3}  + \frac{2}{v_\symup{tr}^3} \right) \symup{d} \omega
  \label{eq:Freq}
\end{equation}
bestimmen. Wobei $v_\symup{l}$ und $v_\symup{tr}$ die Phasengeschwindigkeiten in longitudinaler bzw. transversaler
Richtung beschreiben.\\
Da ein $\symup{N_\symup{L}}$-Atomiger, endlicher Kristall genau 3$\symup{N_\symup{L}}$ Eigenschwingungen hat, ergibt sich
eine obere Grenzfrequenz
\begin{equation}
  \omega_\symup{D}= \left(\frac{18 \pi^2 \symup{N}_\symup{L}}{L^3}
   \left(\frac{1}{v_\symup{l}^3} + \frac{2}{v_\symup{tr}^3}\right)^{-1}\right)
   ^{\frac{1}{3}} ,
\label{eq:DebFre}
\end{equation}
welche Debye-Frequenz genannt wird.\\
Damit ergibt sich für die Molwärme
\begin{equation}
  C_{V,\symup{Debye}} = 9 \symup{R} \left(\frac{T}{\theta_\symup{D}}\right)^3 \int_0^{\frac{\theta_\symup{D}}{T}}
  \frac{x^4  \exp{x}}{(\exp{x} - 1)^2} \symup{d}x
  \label{eq:MolDeb}
\end{equation}
mit $\theta_\symup{D}$ als Debye-Temperatur, welche materialabhängig ist. Gleichung \ref{eq:MolDeb}
zeigt für den Grenzfall hoher Temperaturen
\begin{equation*}
  \lim \limits_{T \to \infty}{{C_{V,\symup{Debye}}}} = 3\symup{R}
\end{equation*}
den zu erwartenden Grenzwert.
Für geringe Temperaturen $T \ll \theta_\symup{D}$ ergibt sich:
\begin{equation*}
  \lim \limits_{T \to 0}{{C_{V,\symup{Debye}}}} \propto T^3 .
\end{equation*}
Dieses Verhalten spiegelt den Verlauf der experimentellen Kurve auch für kleine Temperaturen
ausreichend gut wieder.
