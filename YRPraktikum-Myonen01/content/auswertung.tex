\section{Auswertung}
\label{sec:Auswertung}

% \begin{figure}
%   \centering
%   \includegraphics{plots/plot.pdf}
%   \caption{Plot.}
%   \label{fig:plot}
% \end{figure}



% \begin{table}
%    % Notation :  {% nicht entfernen ist sehr wichtig sonst Fehler !!
% \parbox{0.48\textwidth}{% %Ermöglicht zwei Tabellen neben einander
%   \centering
%   \sisetup{round-mode = places , round-precision = 0,scientific-notation=fixed, fixed-exponent = 0}
%          %rundet Werte aus Stelle, Stelle = ,  macht einen bestimmten festen exponenten
%   \resizebox{\textwidth}{!}{%  % skaliert zu große Tabellen
%   \begin{tabular}{S@{${}\pm{}$} S} % fügt plus minus Fehler Schreibweise hinzu
%     \toprule
%      $\text{e}_b / \si{\milli\meter}$ &
%      $\text{d}_b /\si{\milli\meter} $ & $\text{f}_b / \si{\milli\meter} $\\
%     \midrule
%     \bottomrule
%   \end{tabular}
%   % }
%   \caption{Tabellenunterschrift}
%   \label{tab:tab}
% }
% % \end{table}
% % \begin{table}
% \parbox{0.48\textwidth}{%
%   \centering
%   \sisetup{round-mode = places , round-precision = 0,scientific-notation=fixed, fixed-exponent = 0}
%   % \resizebox{\textwidth}{!}{%
%   \begin{tabular}{S@{${}\pm{}$} S}
%     \toprule
%      $\text{e}_b / \si{\milli\meter}$ &
%      $\text{d}_b /\si{\milli\meter} $ & $\text{f}_b / \si{\milli\meter} $\\
%     \midrule
%     \bottomrule
%   \end{tabular}
%   % }
%   \caption{Tabellenunterschrift}
%   \label{tab:tab}
% }
% \end{table}
Die im folgenden dargestellten Ausgleichs- und Fehlerrechnungen sowie die Diagramme wurden mit den Paketen Numpy \cite{numpy}, Uncertainties \cite{uncertainties}, Matplotlib \cite{matplotlib} und Scipy \cite{scipy} in der Programmierumgebung Pyhton erstellt bzw. durchgeführt.
\subsection{Bestimmung der Auflösungszeit}
Um die optimale Verzögerungszeit zu finden, wurde  $T_\text{Vz}$ varriert und gegen die Anzahl an Startimpulsen pro 10\,s aufgetragen.\\
In Abb. \ref{fig:Verzoegerung} sind die Messwerte aus Tabelle \ref{tab:Verzoegerung} aufgetragen. Da kein scharfes Maximum zu erwarten war,
wurde ein Plateau gefittet und die Halbwertsbreibe bestimmt.\\
Die Fehler der Counts wird mit dem $\sqrt{N}$-Gesetz abgeschätzt.\\
Die maximale Anzahl an Counts $N_{\text{max}}=(232\pm15)$ wurde bei einer Verzögerungszeit von $T_\text{Vz}=4\,\text{ns}$ gemessen, weshalb diese
an der Verzögerungsleitung eingestellt wurde.
\begin{figure}
  \centering
  \includegraphics[width=0.8\textwidth]{plots/plotVerzoegerung.pdf}
  \caption{Messwerte und Fit des Plateaus zur Bestimmung der Verzögerungszeit.}
  \label{fig:Verzoegerung}
\end{figure}
Die in Abb. \ref{fig:Verzoegerung} bestimmte Halbwertsbreite entspricht der Toleranzzeit $\Delta t_{\text{k}}$
der Koinzidenzeinheit und ergibt sich aus den Längen der Diskriminatorimpulsen. Diese Zeit,
 bildet die untere Schranke, dass zwei eingehende Rechteckimpulse aus dem Diskriminator als ein Signal an die Logikschaltung
 weiter gegeben wird. \\
Hierbei sollte die Breite des Plateaus der doppelten Breite der Diskriminatorimpulsen entsprechen.\\
Die Halbwertsbreibe wurde Anhand
des Plots zu
\begin{equation*}
  \Delta t_{\text{k}}=(|-9|+|17|)\,\text{ns}=26\,\text{ns}
\end{equation*}
bestimmt.
\begin{table}
\centering
\caption{Messwerte zur Bestimmung der Auflösungszeit der Messapperatur}
\label{tab:Verzoegerung}
\begin{tabular}{c|c}
$T_{\text{VZ}}$ in ns& $N$\\
\hline
-12& 41\\
-9& 114\\
-6& 167\\
-5& 192\\
-4& 180\\
-3& 192\\
-2& 227\\
-1& 210\\
0& 203\\
0,5& 203\\
1& 242\\
1,5& 214\\
2& 203\\
3& 216\\
3,5& 205\\
4& 232\\
5& 213\\
6& 202\\
7& 197\\
8& 208\\
9& 205\\
10& 198\\
11& 190\\
12& 202\\
13& 197\\
14& 190\\
15& 201\\
16& 151\\
17& 136\\
18& 169\\
19& 83\\
20& 36\\
\end{tabular}
\end{table}
\subsection{Zeitkalibrierung der Messapperatur}
Um den Vielkanalanalysator zu kalibrieren werden mit Hilfe eines Doppelimpuls-Generators Impulse mit unterschiedlichem
zeitlichem Abstand eingelesen. Die belegten Kanäle sind in Tabelle \ref{tab:Kalibrierung} dargestellt und in Abbildung \ref{fig:Kalibrierung}
gegen die Impulsabstände aufgetragen.
\begin{table}[H]
\centering
\caption{Messwerte zur Bestimmung der Zeitkalibrierung der Messapperatur}
\label{tab:Kalibrierung}
\begin{tabular}{c|c}
Kanal&$T_{\text{VZ}}$ in $\upmu$s\\
\hline
23&	0.5\\
45&	1.0\\
68&	1.5\\
90&	2.0\\
112&	2.5\\
135&	3.0\\
157&	3.5\\
179&	4.0\\
202&	4.5\\
224&	5.0\\
246&	5.5\\
269&	6\\
291&	6.5\\
313&	7\\
336&	7.5\\
358&	8\\
381&	8.5\\
402&	9\\
443&	9.9\\
\end{tabular}
\end{table}
\begin{figure}
  \centering
  \includegraphics[width=0.8\textwidth]{plots/plotKanal.pdf}
  \caption{Messdaten und Fit zur Zeitkalibrierungder Apperatur.}
  \label{fig:Kalibrierung}
\end{figure}
Die lineare Regression der Form $f(x)=a\cdot x+b$ wird mit \cite{scipy} bestimmt und liefert für den Zusammenhang von Kanalnummer und Zeitabstand
\begin{equation}
  \Delta t =\left(0,02238 \pm 1,4\times 10^{-5}\right)\,\upmu\text{s}\cdot K + \left(-0,014 ± 0,003\right)\,\upmu\text{s}.
  \label{eq:KanalZeit}
\end{equation}
\subsection{Statistische Berechnung des Untergrundes}
Um den Untergrund zu bestimmen wird die Startimpulsrate
\begin{equation}
  n=\frac{N_{\text{Start}}}{t_{\text{Gesamt}}}=(25,05 \pm 0,01)\,\frac{1}{\symup{s}}
\end{equation}
aus der gesamten Messdauer $t_{\text{Gesamt}}= 83930$\,s und der Anzahl an Startimpulsen $N_{\text{Start}}= (2102553 \pm 1450)$ bestimmt.\\
Mit der poissionverteilten Wahrscheinlichkeit, dass n Teilchen während der Suchzeit $T_{\text{S}}= 11,2\,\upmu\text{s}$ den Tank
durchqueren kann die Anzahl an Fehlmessungen $N_{\text{F}}$ mit
\begin{equation}
  N_{\text{F}}=P\cdot N_{\text{Start}}=\frac{n\cdot T_{\text{S}}}{1!}e^{n\cdot T_{\text{S}}}\cdot N_{\text{Start}}
\end{equation}
berechnet werden.\\
Da diese Werte statistisch gleichverteilt sind kann die Untergrundrate pro Kanal mit
 \begin{equation}
   U=\frac{N_{\text{F}}}{512}
 \end{equation}
bestimmt werden.\\
Damit ergeben sich folgende Werte:
\begin{align*}
  P&=(0,0002805\pm0,0000002)\%\\
  N_{\text{F}}&=(589,8\pm0,8)\\
  U&=(1,152\pm0,002)\frac{1}{\text{Kanal}}
\end{align*}
\subsection{Bestimmung der Lebensdauer der Myonen}
Zur Bestimmung der Lebensdauer $\tau$ der Myonen wird die Anzahl an Stoppsignalen in den Kanalen des Vielkanalanalysators gespeichert.
In Abbildung \ref{fig:KanalCount} sind die Counts $N$ gegen die Kanäle aufgetragen.
\begin{figure}
  \centering
  \includegraphics[width=0.8\textwidth]{plots/KanalCounts.pdf}
  \caption{Messdaten des Vielkanalanalysators.}
  \label{fig:KanalCount}
\end{figure}
Mit Gleichung \ref{eq:KanalZeit} wird aus den belegten Kanälen die Zeitdifferenz $\Delta t$ bestimmt und in Abbildung
\ref{fig:ZeitCounts} sowie \ref{fig:ZeitCountslin} aufgetragen.
\begin{figure}
  \centering
  \includegraphics[width=0.8\textwidth]{plots/ZeitCounts.pdf}
  \caption{Zur Bestimmung der Lebensdauer wurden die gemessenen Counts gegen den berechneten
    zeitlichen Abstand aufgetragen und mit einer Exponentialfunktion gefittet.}
  \label{fig:ZeitCounts}
\end{figure}
\begin{figure}
  \centering
  \includegraphics[width=0.8\textwidth]{plots/ZeitCountslinear.pdf}
  \caption{Der Übersichlichkeit halber wurden die Messdaten und der Fit aus Abb. \ref{fig:ZeitCounts} mit logarithmischer y-Achse dargestellt.}
  \label{fig:ZeitCountslin}
\end{figure}
Die Werte aus Abb. \ref{fig:ZeitCounts} wurden mit der Funktion
\begin{equation}
  N=N_0\cdot e^{-\lambda t}+U_{\text{fit}}
\end{equation}
gefittet.
Dabei ergeb sich für die Parameter folgende Werte:
\begin{align*}
  N_0&=(163 \pm 3)\\
  \lambda&=(0,495 \pm 0,009)\,\frac{1}{\upmu\text{s}}\\
  U_{\text{fit}}&=(0,8 \pm 0,2) \frac{1}{\text{Kanal}}
\end{align*}
Daraus lässt sich die Lebensdauer mit $\tau=\frac{1}{\lambda}$ zu
\begin{equation*}
  \tau=(2,02\pm0,04)\,\upmu\text{s}
\end{equation*}
bestimmen.
