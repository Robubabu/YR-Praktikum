\section{Diskussion}
\label{sec:Diskussion}
Bei der Betrachtung der Untergrundwerte fällt auf, das diese sich leicht von einander unterscheiden.
Die Werte
\begin{align*}
  U_{\text{stat}}&=(1,1519\pm0,0016)\frac{1}{\text{Kanal}}\\
  U_{\text{fit}}&=(0,8 \pm 0,2) \frac{1}{\text{Kanal}}
\end{align*}
zeigen einen relativen Fehler von $(26\pm19)\%$ auf.\\
Mögliche Fehler bei der Bestimmung des Untergrundes können durch die statistische Auswertung entstehen, da in der Berechnung der
Poissonverteilung fehlerbehaftete Größen benutzt wurden.\\
Der Theoriewert für die Lebensdauer der Myonen beträgt $\tau_{\text{theo}}=(2,1969811\pm0,0000022)\,\upmu\text{s}$\cite{Lebensdauer}.
Die relative Abweichung zu dem von uns bestimmten Wert $\tau=(2,02\pm0,04)\,\upmu\text{s}$ beträgt $(8,1\pm1,7)\%$.
Durch die unternommenen Schritte zur Rauschunterdrückung können in mehreren Stellen echte Signale rausgefiltert worden sein.
So kann durch eine zu hoch eingeregelte Diskriminatorschwelle verhindern, dass alle Signale aufgenommen werden.\\
Die Auflösungszeit der Koinzidenzeinheit konnte nur abgeschätzt werden da gerade im Bereich außerhalb des Plateaus
in Abbildung \ref{fig:Verzoegerung} nicht genung Messwerte zu finden waren.
Dadurch, dass zusätzlich die Breiten der Diskriminatorimpulse bei den Messergebnissen fehlen,
kann der Wert von $\Delta t_\text{k}$ auch nicht in Verbindung der Breiten gesetzt werden.
Qualitativ wäre zu erwarten gewesen, dass ein breiterer Diskriminatorimpuls dazu führt, dass das Plateau
breiter wird und somit eine höhere Halbwertsbreite auftritt.\\
Auch durch die eingestellte Suchzeit können Signale verpasst werden die ausserhalb dieser Zeit liegen und damit eine größere
Lebensdauer besitzen.\\
Eine weitere Fehlerquelle ist die Tatsache, dass stark abgebremste negative Myonen nicht zerfallen
sondern von einem Atomkern eingefangen werden kann.\\
Außerdem können auch durch mögliche Unsicherheiten in den elektrischen Bauteilen Verluste auftreten.\\
