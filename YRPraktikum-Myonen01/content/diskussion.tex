\section{Diskussion}
\label{sec:Diskussion}
Bei der Betrachtung der Untergrundwerte fällt auf, das diese sich leicht von einander unterscheiden.
Die Werte
\begin{align*}
  U_{\text{stat}}&=(1,1519\pm0,0016)\frac{1}{\text{Kanal}}\\
  U_{\text{fit}}&=(0,8 \pm 0,2) \frac{1}{\text{Kanal}}
\end{align*}
zeigen einen relativen Fehler von $(26\pm19)\%$ auf.\\
Der Theoriewert für die Lebensdauer der Myonen beträgt $\tau_{\text{theo}}=(2,1969811\pm0,0000022)\,\upmu\text{s}$\cite{Lebensdauer}.
Die relative Abweichung zu dem von uns bestimmten Wert $\tau=(2,02\pm0,04)\,\upmu\text{s}$ beträgt $(8,1\pm1,7)\%$.
Die Fehler können hier nicht weiter disskutiert werden, da die benutzten Werte nicht von uns aufgenommen wurden.\\
Mögliche Fehler bei der Untergrund Bestimmung können aber durch die statistische Auswertung entstehen, da in der Berechnung der
Poissonverteilung fehlerbehaftete Größen benutzt wurden.\\
Durch die eingestellte Suchzeit können Signale verpasst werden die knapp ausserhalb dieser Zeit liegen.\\
 Außerdem können auch durch mögliche geräteinternen Unsicherheiten Verluste auftreten.\\
