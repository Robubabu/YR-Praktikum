\section{Durchführung}
\label{sec:Durchführung}
\subsection{Aufbau, Prüfung und einstellen des Versuchsaufbaus.}
Zuerst vergewissert man sich, unter Zuhilfenahme eines Oszillographen, dass die SEV funktionieren. 
Auf dem Oszillographen sollten dann zufällige Impulse zu betrachten sein, dabei ist zu beachten 
einen \SI{50}{\ohm} Widerstand parallel zuschalten um Reflexionen zu vermeiden. 
Danach schließt man die SEV an die Diskriminatoren an und gibt die Signale der Diskriminatoren
auf den Oszillographen. Es sollten Impulse mit gleicher Höhe und Länge zusehen sein, ist dies nicht 
der Fall muss dies über den Diskriminator eingestellt werden. Nun kann an Stelle des Oszillographen 
ein Zählwerk angeschlossen werden, beide Diskriminatoren sollten ungefähr die selbe Anzahl an 
Impulsen ausgeben. Ist dies nicht der Fall muss die Schwelle des Diskriminators eingeregelt werden. 
Nun wird die Koinzidenzschaltung und die Verzögerung angeschlossen. Das Signal der 
Koinzidenzschaltung wird nun auf das Zählwerk gegeben und die Verzögerung variiert. Die Zählrate 
wird gegen die Verzögerung aufgetragen. Es ergibt sich ein "Plateau" die Verzögerung sollte ungefähr 
auf die Mitte dieses Plateaus eingestellt sein. Jetzt sollte noch die Zählrate nach und vor 
der Koinzidenzschaltung verglichen werden. Ist kein Unterschied zuerkennen sollte die Schwelle des 
Diskriminators abgesenkt werden. 
\newline
Der Teil des Aufbaus der bis jetzt aufgebaut ist, wird nun abgeklemmt und ein Doppelimpulsgenerator 
wird an die Koinzidenzschaltung angeschlossen. Nun wird die Verzögerung und der Univibrator 
angeschlossen. Jetzt kann mit einem Oszillographen die Messzeit $T_s$ gemessen werden, diese sollte 
nicht unwesentlich größer sein als die Messzeit des TAC. Jetzt können die AND-Gatter angeschlossen 
werden, wie in Abbildung \ref{fig:aufbau} dargestellt. Mit einem Oszillographen wird jetzt 
überprüft, ob auch die Signale die in den TAC hineingehen den vom Doppelimpulsgenerator generierten 
Abstand haben. Die Messzeit am TAC wurde bei dem verwendeten Versuchsaufbau auf \SI{20}{\micro\second} 
eingestellt.
\newline 
Jetzt wird noch notiert, welcher Kanal des Vielkanalanalysators welcher Zeit entspricht. Danach kann 
die Versuchsaufbau, zusammengebaut werden wie in \ref{fig:aufbau} dargestellt. Die Messung kann nun 
begonnen werden.   
\subsection{Messung.} 
Die Messung beginnt, wenn das Zählwerk und die Aufzeichnung des Vielkanalanalysators gleichzeitig 
gestartet wird. Ab dem Zeitpunkt wird dann 20 bis \SI{30}{\hour} gemessen. 
Danach wird wieder das Zählwerk und die Aufzeichnung gleichzeitig gestoppt. Eine Messreihe besteht 
aus den Aufzeichnungen des Vielkanalanalysators, der Anzahl der detektierten Myonen, der Anzahl 
der Fehlmessungen und der Messzeit. 
