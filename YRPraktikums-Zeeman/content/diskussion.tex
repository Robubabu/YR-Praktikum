\section{Diskussion}
\label{sec:Diskussion}
Alle Messergebnisse und ihre Literaturwerte, die mit den Formeln \eqref{eq:Lande} und 
\eqref{eq:Egij}, 
bestimmt wurden, sind in der Tabelle \ref{tab:D} zusammengefasst. Die Abweichung können 
durch die Messungen des Magnetfeldes und den Auswertungsmethoden aus Kapitel \ref{sec:MB} erklärt 
werden. Zur Messung des Magnetfeldes ist zu diskutieren, dass die Messung nicht an der Stelle der 
Lampe geschehen ist, sondern an den Polschuhen des Magneten. Daraus folgt, dass die gemessen Werte 
nicht genau den Bereich beschreiben, in dem das Magnetfeld wirkt und somit die tatsächlichen Werte 
von den Messwerten abweichen können. Zu den Auswertungsmethoden ist zu diskutieren, dass diese 
ebenfalls einem Fehler unterliegen, da die Intensitätsmaxima nur bis zu einem gewissen Grad genau 
zu lokalisieren sind, da Umgebungs- und Kameraeffekte die genaue Position der Maxima 
verlaufen lassen. 
\begin{table}
 \centering
 \begin{tabular}{S S S S@{$\quad \pm \quad$} S S} 
  \toprule
  $\text{Wellenlänge} / \si{\nano\meter} $ 
  & $\text{Polarisation}$
  & \multicolumn{3}{c}{$\text{Übergangs Lande-Faktor}$} 
  & $ \text{Relative Abweichung} / \si{\percent}$ \\
  \midrule
  & &$\text{theoretisch} $&\multicolumn{2}{c}{$\text{experimentell}$}& \\
  \midrule
  643.8 & $\sigma$ & 1 &1& 0.026 &0 \\  
  480 & $\sigma$ & 1.75 & 1.41 & 0.13 & 20 \\
  480 & $  \text{\texorpdfstring{$\pi$}{math}}$ & 0.5 & 0.41 & 0.06 & 18 \\
  \bottomrule
  \end{tabular}
 \caption{Messergebnisse und deren Erwartungswerte, sowie relativen Abweichungen im Überblick.}
 \label{tab:D}
\end{table}
