\section{Theorie}
\label{sec:Theorie}
\subsection{Das magnetische Moment}
Das Elektron hat neben dem Bahndrehimpuls $\vec{l}$ einen weiteren Drehimpuls den Spin $\vec{s}$.\\
Durch die beschleunigte Ladung des Elektron ergibt sich ein Zusammenhang mit dem magnetische Moment. So lässt sich das
Bohrsche Magneton
\begin{equation}
  \symup{\mu_\symup{B}} \coloneq -\frac{1}{2}\hbar\frac{\symup{e_0}}{\symup{m_0}}
\end{equation}
definieren.\\
Damit lassen sich die magnetischen Momente für den Spin $\mu_s$ und den Bahndrehimpuls $\mu_l$ wie folgt berechnen:
\begin{align}
  \vec{\mu_l} &= - \symup{\mu_\symup{B}}\sqrt{l(l+1)}\vec{e}_l\\
  \vec{\mu_s} &=-\symup{g_\symup{s}} \symup{\mu_\symup{B}}\sqrt{s(s+1)}\vec{e}_s \label{eq:muspin}
\end{align}
In \ref{eq:muspin} wird der sogenannte Landé-Faktor $\symup{g_\symup{s}}\approx 2$ eigeführt.
\subsection{Wechselwirkung der Drehimpulse}
Bei Atomen muss unterschieden werden, ob die Bahndrehimpulse und magnetischen Momente der einzelnen Elektronen
 untereinander wechselwirken oder diese zwischen den verschiedenen Elektronen in Wechselwirkung treten.\\
Eine vereinfachende Einteilung gelingt über die Kernladungszahl $Z$ der Atome.
Für Atome mit hoher Kernladungszahl ist die Wechselwirkung zwischen Spin und Bahndrehimpuls des Einzelelektrons
so groß, dass sich die Komponenten wie folgt zusammensetzten:
\begin{equation}
  \vec{j}_i= \vec{l}_i+ \vec{s}_i .
\end{equation}
Damit lässt sich ein Gesamtdrehimpuls
\begin{equation}
  \vec{J}=\sum_{i=1}^{Z} \vec{j}_i
\end{equation}
berechnen. Dies wird als j-j-Kopplung bezeichnet. Zwischen diesem Spezialfall und dem im
 Folgenden beschriebenen besteht ein fließender Übergang.\\
\\Bei niedriger Kernladungszahl lassen sich die Bahndrehimpulse zu einem Gesamtdrehimpuls
\begin{equation}
  \vec{L}=\sum_{i=1}^{Z} \vec{l}_i , \qquad \text{mit } |{\vec{L}}| =\sqrt{L(L+1)}\hbar
\end{equation}
zusammenfassen.
Dasselbe gilt für den Gesamtspin
\begin{equation}
  \vec{S}=\sum_{i=1}^{Z} \vec{s}_i , \qquad \text{mit } |{\vec{S}}|=\sqrt{S(S+1)}\hbar.
\end{equation}
Dazu gehören die folgenden magnetische Momente:
\begin{align*}
  |{\vec{\mu}}_\symup{L}|&=\symup{\mu_B}\sqrt{L(L+1)}\\
  |{\vec{\mu}}_\symup{S}|&=\symup{g_\symup{S}}\symup{\mu_B}\sqrt{S(S+1)}.
\end{align*}
Die Summendrehimpulse $\vec{L}$ und $\vec{S}$ lassen sich zu einem Gesamtdrehimpuls
\begin{equation}
  \vec{J}= \vec{L}+\vec{S} , \qquad \text{mit } |{\vec{J}}|=\sqrt{J(J+1)}\hbar.
\end{equation}
zusammenfassen, wenn kein zu großes externes Magnetfeld anliegt.
Die hier beschriebene LS-Kopplung liegt dem Zeeman-Effekt zugrunde.\\
Die Drehimpulsquantenzahl $J$ kann entweder ganz oder halbzahlig sein, da $L$ ganzzahlig ist und $S$ halbzahlig
sein kann. Mit diesen drei Quantenzahlen lassen sich die Energieniveaus beschreiben. Der Drehimpulsquantenzahl $L$
werden die Bezeichnungen S($L=0$), P($L=1$), D($L=2$) und F($L=3$) und zugeordnent. Die Gesamtdrehimpulsquantenzahl $J$
wird als unterer Index und die Multiplizität $M=2S+1$ oben links an die S,P,D,F - Terme geschrieben.
\subsection{Energieaufspaltung}
Für den mittels LS-Kopplung entstandenen Gesamtdrehimpuls kann nun ein magnetisches Moment
\begin{equation}
  \vec{\mu}=\vec{\mu_\symup{L}}+\vec{\mu_\symup{L}}
\end{equation}
definiert werden.
Dabei ist anzumerken, dass die Richtungen von $\vec{J}$ und $\vec{\mu}$ nicht umbedingt zusammenfallen.\\
Der Landé-Faktor lässt sich mittels
\begin{equation}
  \symup{g_\symup{J}}=\frac{3J(J+1)+S(S+1)-L(L+1)}{2J(J+1)}
  \label{eq:Lande}
\end{equation}
bestimmen.\\
Aus der Quantenmechanik folgt, dass nur solche Winkel zwischen $\vec{\mu}$ und
$\vec{B}$ erlaubt sind, bei denen die Komponenten von $\mu_{J_z}$ in Feldrichtung ein ganzzahliges Vielfaches von
$symup{g_\symup{J}}\symup{\mu_\symup{B}}$ sind. Dies wird Richtungsquantelung genannt.\\
Die Ganzzahligkeit von
\begin{equation}
  \mu_{J_z}=-m\symup{g_J}\symup{\mu_B}.
\end{equation}
 wird durch die Orientierungsquantenzahl
\begin{equation*}
  m=-J,-J+1,\dots,1,\dots, J
\end{equation*}
sichergestellt.\\
Mit diesenen Definitionen ist es nun möglich einen Ausdruck für die vom Magnetfeld übertragene Energie
zu bestimmen:
\begin{align}
  E_{\text{mag}}&=-\vec{\mu}_\symup{J} \cdot {\vec{B}}\\
                &=m\symup{g_J}\symup{\mu_B}\vec{B}.
\end{align}
Dies bedeutet, dass sich das Energieniveau $E_0$ eines Atoms beim Einschalten eines Magnetfeldes
in $2J+1$ äquidistante Niveaus aufspaltet. Die Anzahl der Aufspaltungen folgt gewissen Auswahlregeln.
\subsection{Auswahlregeln}
Zur Herleitung der Auswahlregeln wird mit der Schrödinger-Gleichung
\begin{equation}
  -\frac{\hbar^2}{2m}\Delta \Psi\left(\vec{r},t\right) + U  \Psi\left(\vec{r},t\right) + \frac{\hbar}{\symup{i}} \frac{\partial}{\partial t}  \Psi\left(\vec{r},t\right)=0.
  \label{eq:Schroedinger}
\end{equation}
und einer ebenen Welle
\begin{equation}
  \Psi_i \left(\vec{r},t\right)=\Psi_i \left(\vec{r}=0,t=0\right)\exp\left(-\frac{\symup{i}}{\hbar}E_i t\right).
\end{equation}
als Lösung von \ref{eq:Schroedinger} begonnen.\\
Es wird nun eine Superposition $\Psi_{\text{ges}}$ zweier Lösungen von
 \ref{eq:Schroedinger} mit $i=\alpha$ und $i=\beta$  betrachtet.
Die zeitabhängige Dichteverteilung
\begin{align}
\begin{split}
  \int \Psi^{*}_{\text{ges}}\Psi_{\text{ges}} &=
  \int \Big(
   C^2_{\alpha} \Psi^{*}_{\alpha} \Psi_{\alpha}+
   C^2_{\alpha} \Psi^{*}_{\beta} \Psi_{\beta}
   +C_{\alpha} C_{\beta} \Psi^{*}_{\alpha} \Psi_{\beta} \exp \left(\frac{\symup{i}}{\hbar}\left(E_{\alpha}-E_{beta}\right) t \right) \\
   &+C_{\beta} C_{\alpha} \Psi^{*}_{\beta} \Psi_{\alpha} \exp \left(\frac{\symup{i}}{\hbar}\left(E_{\alpha}-E_{beta}\right) t \right)
   \Big) \symup{dV}
  \label{eq:kackscheiß}
\end{split}
\end{align}
beschreibt das Schwingen eines Elektrons mit der Frequenz
\begin{equation}
  \nu_{\alpha \beta}= \frac{E_{\alpha}-E_{\beta}}{\symup{h}} .
\end{equation}
Es lässt sich also ein Ausdruck für einen beliebig Orientierten Dipol bestimmen. Für die Raumkompnenten $i=x,y,z$
ergibt sich:
\begin{equation}
  D_i=-\symup{e}_0 \symup{const} 2 \symup{Re}\left(\int i \Psi^{*}_{\beta} \Psi_{\alpha} \symup{dV} \exp \left( 2 \pi \symup{i} \nu_{\alpha \beta} t \right) \right)
\label{eq:Dipol}
\end{equation}
Die Integrale
\begin{equation*}
  i_{\alpha \beta}= \int i \Psi^{*}_{\beta} \Psi_{\alpha} \symup{dV}
\end{equation*}
beschreiben als Matrixelemente die Strahlungsemission.
Das betrachtete Magnetfeld soll in z-Richtung liegen.
 Dann ergibt sich mit dem Ansatz für die Wellenfunktion eines Atoms
 $\psi=\frac{1}{\sqrt{2 \pi}}R(r)\Theta(\theta)\exp(\symup{i}m\phi)$ im Magnetfeld für
\begin{equation}
  z_{\alpha \beta}=\frac{1}{2\pi}\int_0^{\infty} R_{\alpha}(r)R_{\beta}(r)r^3\symup{dr}
  \int_0^{\pi} \Theta_{\alpha} \Theta_{\beta} \sin \theta \cos \theta \symup{d\theta}
  \int_0^{2\pi} \exp\left(\symup{i}\left(m_{\alpha}-m_{\beta}\right)\phi \right) \symup{d\phi} .
\label{eq:Dipolz}
\end{equation}
Damit in \ref{eq:Dipolz} das letzte Integral nicht verschwindet muss $m_{\alpha}=m_{\beta}$ gelten.
Die Orientierungsquantenzahl muss also bei beiden Zuständen gleich sein.\\
Der Dipol schwingt also in Feldrichtung und seine stärkste Emission ist senkrecht dazu. Die Strahlung
ist also linear in $\vec{B}$-Richtung polarisiert. Diese Übergänge werden als $\pi$-Komponente des Zeeman-Tripletts bezeichnet.\\
\\Für die x- und y-Komponenten ergeben sich zu \ref{eq:Dipolz} analoge Terme . Betrachtet man die Gleichungen
\begin{equation}
    x_{\alpha \beta}\pm \symup{i} y_{\alpha \beta}=
    \frac{1}{2\pi}\int_0^{\infty} R_{\alpha}(r)R_{\beta}(r)r^3\symup{dr}
    \int_0^{\pi} \Theta_{\alpha} \Theta_{\beta} \sin^2 \theta \symup{d\theta}
    \int_0^{2\pi} \exp\left(\symup{i}\left(m_{\alpha}-m_{\beta} \pm 1\right)\phi \right) \symup{d\phi}
\end{equation}
erkennt man mit der selben Begründung wie oben, dass die weiteren Auswahlregeln durch
\begin{align*}
  m_{\beta}&=m_{\alpha}+1\\
  m_{\beta}&=m_{\alpha}-1
\end{align*}
gegen sind.\\
Damit ergibt sich, dass $x_{\alpha \beta}$ und $y_{\alpha \beta}$ bis auf ein eine Phasenverschiebung von $\pm \frac{\pi}{2}$ gleich sind.
Es tritt also eine zirkular-polarisierte Schwingung auf. Diese Übergänge werden als $\sigma$-Komponenten des Zeeman-Tripletts bezeichnet.
\subsection{Der normale und annormale Zeeman-Effekt}
Aus historischen Gründen wird der Spezialfall $S=0$ \textbf{normaler Zemann-Effekt} genannt.\\
Für den Landé-Faktor gilt in diesem Fall nach \ref{eq:Lande} $\symup{g_\symup{J}}=1$ also ist die Aufspaltung Energieniveaus immer
\begin{equation}
  \Delta E=m\symup{\mu_{B}}B, \qquad \text{mit } -J \leq m \leq J .
\end{equation}
  Der \textbf{annormale Zeeman-Effekt} tritt auf, wenn der Elektronenspin mitbetrachtet wird, also wenn $S\neq 0$ gilt.
  Damit hängt der Landé-Faktor von L,S und J ab. Die Energieaufspaltung wird dann mit
  \begin{equation}
    \Delta E=\symup{g}_{ij} \symup{\mu_{B}}B, \qquad \text{mit } \symup{g}_{ij}=m_i\symup{g}_{i}- m_j\symup{g}_{j}.
  \end{equation}
  berechnet.
