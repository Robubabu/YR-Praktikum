\section{Diskussion}
\label{sec:Diskussion}
Alle relativen Fehler wurden nach der Formel
\begin{equation*}
  \tilde{x} = \frac{ \lvert x_{lit} - x_{mess} \rvert}{\lvert x_{lit} \rvert}
  \cdot 100 \%
\end{equation*}
berechnet, dabei bezeichnet $x_{lit}$ den Literaturwert der Messgröße $x_{mess}$.\\
An den Abbildungen \ref{fig:SaugDreh} und \ref{fig:SaugTurbo} ist zu erkennen, dass bei der Turbomolekularpumpe eine
deutliche Abweichung von den bestimmten Saugvermögen zu dem vom Hersteller angegeben Wert auftritt. Bei der Drehschieberpumpe
ist die Abweichung geringer.
In der Tabelle \ref{tab:Ergeb} sind die Ergebnisse nochmal zusammengefasst.
\begin{table}[H]
\centering
\caption{Zusammenfassung der Ergebnisse zum Saugvermögen}
\label{tab:Ergeb}
\begin{tabular}{c|c|c}
  & Drehschieberpumpe &Turbomolekularpumpe\\
  \toprule
 $S_\symup{theo}\,\frac{\symup{l}}{\symup{s}}$  & 1,1 & 77,0\\
  $S_\symup{g}\,\frac{\symup{l}}{\symup{s}}$    &$(0,9 \pm 0,07)$&$(11,9 \pm 0,9)$ \\
  $S_\symup{lin1}\,\frac{\symup{l}}{\symup{s}}$ &$(1,05 \pm 0,08)$ &$(12,2 \pm 0,9)$ \\
 $S_\symup{lin2}\,\frac{\symup{l}}{\symup{s}}$  &$(0,73 \pm 0,07) $&$(9,1 \pm 0,7) $\\
 $S_\symup{lin3}\,\frac{\symup{l}}{\symup{s}}$  &$(0,24 \pm 0,04) $& - \\
 $S_\symup{leck1}\,\frac{\symup{l}}{\symup{s}}$ &$(0,6 \pm 0,1) $&$ (25 \pm 3) $\\
$S_\symup{leck2}\,\frac{\symup{l}}{\symup{s}}$  &$(0,8 \pm 0,2) $&$(22 \pm 3) $\\
 $S_\symup{leck3}\,\frac{\symup{l}}{\symup{s}}$ &$(0,8 \pm 0,2) $&$ (12 \pm 2) $\\
  $S_\symup{leck4}\,\frac{\symup{l}}{\symup{s}}$ &$(0,8 \pm 0,2) $&$ (20 \pm 3) $\\
\bottomrule
\end{tabular}
\end{table}
Der geringste relative Fehler zum Literaturwert der Drehschieberpumpe hat $S_\symup{lin1}$ mit
$(5\pm7)\%$.\\
Für die Turbomolekularpumpe zeigt $S_\symup{leck1}$ mit einer Abweichung von $(68 \pm 4)\%$
den geringsten Unterschied.\\
Diese Unterschiede zu den Herstellerangaben lassen sich dadurch erklären, dass
diese unter idealen Begebenheiten gemessen werden, welche bei diesem Versuch zumindest
für die Turbomolekularpumpe nicht gegeben sind.\\
Ein Problem besteht darin, dass das Rohr welches an die Pumpe angeflanscht wird,
einen kleineren Durchmesser besitzt, wie in der Abbildung \ref{fig:Aufbau} zu erkennen ist. Dies sorgt für
einen größeren Strömungswiderstand was zu einem vermindertem Saugvermögen führt.
Bei der Drehschieberpumpe ist die Differenz der beiden Durchmesser klein. Somit wird die Drehschieberpumpe unter besseren Bedingungen
betrieben, was sich auch in den Unterschieden zwischen dem experimentell bestimmten Wert für das Saugvermögen
und der Herstellerangabe wieder spiegelt.\\
Desweiteren ist zu bemängeln, dass das Volumen des Rezipienten relativ gering ist. Ein größeres Volumen würde für ein
aussagekräftigeres Ergebnis sorgen.\\
Es muss beachtet werden, dass trotz längerem Heizen der Apparatur vor dem Versuch, immer noch Wasseransammlungen an einer bestimmten Stelle
des Tanks vermutet werden, da ein deutlicher Druckanstieg bei Erwärmung an einer bestimmten Stelle zu erkennen war.
Deswegen werden Desorptionsvorgänge vermutet.\\
Bei den Leckratenmessungen muss erwähnt werden, dass die Ventile per Hand geschlossen wurden und es nicht auszuschließen ist,
dass der Schließvorgang unterschiedlich lange gedauert hat oder das Ventil nicht immer komplett dicht war. Dadurch lassen
sich bei einigen Messreihen, zum Beispiel in \ref{tab:leck_turbo3}, die deutlichen Abweichungen unter den Zeitmessungen erklären.\\
Bei der Zeitmessung könnte durch eine etwaige Automatisierung verbessert werden, da gerade das gleichzeitige Ablesen und Umschalten der
Messgeräte zu Fehlern führen kann.
Abschließend lässt sich sagen, dass der Versuchsaufbau eine gute Möglichkeit für die Untersuchung der Drehschieberpumpe bietet.
Für die Turbomolekularpumpe könnte es zu besseren Ergebnissen führen, wenn der Aufbau so abgeändert wird, dass keine Querschnittsverengungen
der Rohre vorliegen würden.
