\section{Diskussion}
\label{sec:Diskussion}
An den Abbildungen \ref{fig:SaugDreh} und \ref{fig:SaugTurbo} ist zu erkennen, dass das Saugvermögen druckabhängig ist.
Desweiteren zeigt sich, dass bei der Turbomolekularpumpe eine
deutliche Abweichung von den bestimmten Saugvermögen zu dem vom Hersteller angegeben Wert von $77\,\frac{\symup{l}}{\symup{s}}$ auftritt.
Bei der Drehschieberpumpe ist die Abweichung zum angegebenen Wert von $1,1\,\frac{\symup{l}}{\symup{s}}$ geringer.\\
Diese Unterschiede zu den Herstellerangaben lassen sich dadurch erklären, dass
diese unter idealen Begebenheiten gemessen werden und nur der maximale Wert als angegeben wird.
Für die Drehschieberpumpe zeigt sich, dass der höchste von uns bestimmte Wert im Rahmen der Messfehler den Herstellerwert
reproduzieren konnte.\\
Dies gilt jedoch nicht für die Turbomolekularpumpe, was an folgenden Problemen im Versuchsaufbau liegen kann.
Das Rohr, welches an die Pumpe angeflanscht wird,
hat einen kleineren Durchmesser, wie in der Abbildung \ref{fig:Aufbau} zu erkennen ist. Dies sorgt für
einen größeren Strömungswiderstand was zu einem vermindertem effektivem Saugvermögen führt.
Bei der Drehschieberpumpe ist die Differenz der beiden Durchmesser klein. Somit wird die Drehschieberpumpe unter besseren Bedingungen
betrieben, was sich auch in den Unterschieden zwischen dem experimentell bestimmten Wert für das Saugvermögen
und der Herstellerangabe wieder spiegelt.\\
Desweiteren ist zu bemängeln, dass das Volumen des Rezipienten für die Vermessung der Turbomolekularpumpe zu gering ist. Ein größeres Volumen würde für ein
aussagekräftigeres Ergebnis sorgen.\\
Es muss beachtet werden, dass trotz längerem Heizen der Apparatur vor dem Versuch, immer noch Wasseransammlungen an einer bestimmten Stelle
des Tanks vermutet werden, da ein deutlicher Druckanstieg bei Erwärmung an einer bestimmten Stelle zu erkennen war.
Deswegen werden Desorptionsvorgänge vermutet.\\
Bei den Leckratenmessungen muss erwähnt werden, dass die Ventile per Hand geschlossen wurden und es nicht auszuschließen ist,
dass der Schließvorgang unterschiedlich lange gedauert hat oder das Ventil nicht immer komplett dicht war. Dadurch lassen
sich bei einigen Messreihen, zum Beispiel in \ref{tab:leck_turbo3}, die deutlichen Abweichungen unter den Zeitmessungen erklären.\\
Bei der Zeitmessung könnte durch eine etwaige Automatisierung verbessert werden, da gerade das gleichzeitige Ablesen und Umschalten der
Messgeräte zu Fehlern führen kann.
Abschließend lässt sich sagen, dass der Versuchsaufbau eine gute Möglichkeit für die Untersuchung der Drehschieberpumpe bietet.
Für die Turbomolekularpumpe könnte es zu besseren Ergebnissen führen, wenn der Aufbau so abgeändert wird, dass keine Querschnittsverengungen
der Rohre vorliegen und das Volumen des Tanks vergrößert würden.
