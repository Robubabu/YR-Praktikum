\input{header.tex}
\begin{document}
\section*{FP-Gedächtnisprotokoll - 01.03.18 Prof.Betz} 
\paragraph{Vorwort}
Wir erheben keinen Anspruch auf Vollständigkeit. Alle Fragen und Antworten sind im Gespräch 
entstanden und nicht unbedingt wie dargestellt. Dieses Protokoll soll nur einen generellen Überblick 
über die besprochenen Themen bieten und stellt nur unsere Sicht der Prüfungssituation dar. \\
List der von uns durchgeführten Versuche:

\begin{itemize} 
\item Lebensdauer kosmischer Myonen
\item Faraday-Effekt 
\item Zeeman-Effekt
\item Vakuumphysik
\item Molwärme von Kupfer
 \end{itemize}
\paragraph{Prüfung:} \quad \newline
\textbf{Womit wollen Sie anfangen?}\\
Lebensdauer kosmischer Myonen. \\
\textbf{Was sind Myonen?}\\
Fermionen, Leptonen. \\ 
\textbf{Was gibt es noch für Leptonen?}\\
Elektronen, Tauonen , Neutrinos und deren Antiteilchen. \\
\textbf{Wie sieht das mit Massen und Lebensdauer der Leptonen aus ?} \\
Die Masse vom Myon ist 200mal größer als die des Elektrons und die des Tauons ist noch größer.
Für die Lebensdauern gilt: Das Elektron ist stabil, das Myon zerfällt nach \SI{2.2}{\micro\second} 
das Tauon zerfällt noch schneller. \\
\textbf{Wie sieht denn die Ruheenergie des Elektron aus ? Und warum zerfällt das Tauon so schnell?} \\
Masse des Elektrons ist \SI{511}{\kilo\eV} und das Tauon zerfällt weil es schwerer ist und somit 
mehr Energie freigeben kann. \\
\textbf{Wie entstehen Myonen?}\\
Protonen wechselwirken mit der oberen Atmosphäre und es entstehen Pionen, die wiederum in Myonen 
zerfallen. \\
\textbf{Können Sie den Zerfall aufzeichnen?} \\
(*Zeichne die Reaktionsgleichung Myon zu Elektron und zwei Neutrinos hin*) \\
\textbf{Woher wissen wir, dass wir Neutrinos brauchen und wie groß ist die Energie der Myonen ?} \\
Energie- und Impulserhaltung in den Zerfällen, historisches Bsp. der Beta-Zerfall. 
Die Myonen haben eine Energie von mehreren \SI{100}{\mega\eV}. Sie bewegen sich nahe zu lichtschnell
 und erreichen deshalb innerhalb ihrer Lebensdauer die Erde, da relativistisch Gerechnet werden muss 
(Zeitdilatation).\\
\textbf{Wie detektieren wir unsere Myonen?}\\
In einem Szintillator. Darin befinden sich organische Moleküle die von der kin. Energie der 
Myonen angeregt werden und darauf hin  bei der Abregung einen Lichtimpuls aussenden. \\
\textbf{In welchem Spektrum befinden sich diese Lichtimpulse und wie detektieren wir diese?} \\
Lichtimpulse im sichtbaren Bereich. Sie werden von Photokathoden detektiert, die bei Lichteinfall 
aufgrund des Photoeffekts Elektronen emittieren, dies ist im Infrarotbereich nicht möglich. \\
\textbf{Zeichnen Sie bitte den Versuchsaufbau auf.} \\
(*Zeichnen den Versuchsaufbau auf (Tank und SEV sowie Diskriminatoren) und werden darauf schon 
unterbrochen*) \\
\textbf{Warum benutzen wir zwei SEV ? }\\
Die Photokathoden neigen zu spontanen Elektronenemission. Das ist ein statistische Prozess und 
das dieses Ereignis bei zwei SEV gleichzeitig auftritt ist sehr unwahrscheinlich. \\
\textbf{Kommen wir jetzt zu einem anderen Versuch. Was ist denn der Faraday-Effekt?}\\
Optisch aktive Medien lassen die Polarisationsebene bei Einfall von linearpolarisiertem Licht 
im Medium rotieren. Bei inaktiven Medien kann man diesen Effekt durch anbringen eines äußeren 
Magnetfeldes erreichen. Diesen Effekt nennt man Faraday Effekt. \\
\textbf{Kennen Sie ein Bsp. für ein aktives Medium ? Müssen Sie jetzt auch nicht wissen. Ein Tipp:
 Es kommt nicht in der Festkörperphysik vor.} \\
(*Hier hatten wir keine Ahnung und haben einfach mit Halbwissen um uns geworfen bis er uns die 
Antwort gegeben hat.*) 
\textit{Antwort: Zuckerlösung. Sein Kommentar dazu: Die chemisch-interessierten 
wissen das manchmal.} \\
\textbf{Zeichnen Sie bitte den Aufbau und erklären Sie ,welche Lampe verwendet wurde und wieso ?}\\
(*Zeichen Aufbau*) Wir verwenden eine Halogenlampe im Infrarotbereich. Da unsere Proben nur da 
durch lässig sind. \\
\textbf{In welchen Magnetfeldbereich haben Sie gemessen? Und kommen Sie mit diesem Aufbau bis 5 
Tesla? Was müssen Sie ändern um das zu erreichen?}\\
Wir haben bis zu 0.3 T gemessen. 5 Tesla schaffen wir nicht aufgrund der Erhitzung der Spulen. 
Änderung wären Kühlung oder Supraleitende Materialien. \\
\textbf{Welches Material untersuchen Sie und warum ist es nur bis dahin durchlässig?} \\
Wir benutzen Galliumarsenid. (*Warum es bis dahin durchlässig ist keine Ahnung wir raten 
Interferenz Phänomene*) \\ 
\textbf{Nein, daran liegt es nicht. Denken Sie mal daran, dass GaAs ein Halbleiter ist. Wie sieht 
da die Bandstruktur aus und denken Sie an die Bandlücke.}\\
(*Hier haben wir dann viel hin- und herüberlegt. Bisschen was über das Leitungsband und Valenzband 
geredet. Im Endeffekt lief es darauf hinaus, dass wir darauf gekommen sind das vor dem 
Infrarotspektrum das [bei sichtbarem Licht] ein Elektron aus dem Valenzband in das Leitungsband gehoben wird. *) \\
\textbf{Dann erklären Sie mal den Aufbau und wie wir den Winkel messen.} \\
(*Gehen den Aufbau durch und erklären alle Elemente und Funktionsweisen in groben Zügen*) \\
\textbf{Welche Materialeigenschaften haben wir untersucht?} \\
Wir haben die effektive Masse untersucht. Diese beträgt $0.06\cdot m_e$. \\
\textbf{Kennen Sie Anwendungsbeispiele und so eine geringe Masse von Vorteil ist?} \\
Elektrische Leitfähigkeit. (*Er war zufrieden wollte aber auf Newton 2. und schnelle Vorgänge 
(weil kleine Masse große Beschleunigung) in Festkörpern hinaus.*) \\ 

\paragraph{Fazit}
Wir wollten Lebensdauer kosmischer Myonen vorstellen, da  wir den Versuchsaufbau als interessant und 
lang genug erachtet haben und dachten, dass besonders Frau Siegmann auf die elektronischen 
Bauteile der Schaltung eingehen würde. Entgegen unser Erwartung hat Frau Siegmann nichts gefragt 
und nur Protokoll geführt. Dagegen ist Prof. Betz vollkommen auf den Teilchencharakter dieses 
Versuchs eingegangen (Unser Plan ging also vollkommen nach hinten los). 
Es ist zu bemerken, dass Prof. Betz nach Zahlenwerten (und nicht Größenord.) fragt, wenn nach einer 
Größe gefragt wird. Besonders auffällig ist, dass er sich extrem genau mit den Versuchsaufbauten 
auskennt. \\ 
Abschließend fanden wir die Prüfungsatmosphäre ganz angenehm. Nach 20 min Prüfung und trotz 
kleinerer Aufhänger haben wir eine 1.0 bekommen. 
\end{document}

