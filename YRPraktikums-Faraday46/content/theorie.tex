\section{Theorie}
\label{sec:Theorie}
\subsection{Die effektive Masse}
 Mittels Energiebändern, wie in Abbildung \ref{fig:Energiebänder} zu sehen, lassen sich verschiedene physikalische Effekte
 von Kristallen beschreiben.
 \begin{figure}[H]
 \center
 \includegraphics[width=0.5\textwidth]{pics/Energiebaender.jpg}
 \caption{Schematische Darstellung der Bandstruktur eines Festkörpers \cite{Anleitung}.}
 \label{fig:Energiebänder}
 \end{figure}
 An der Stelle $k=0$ lässt sich die Funktion $\epsilon(\vec{k})$, welche die Elektronenenergie beschreibt, um ihr Minimum entwickeln.
 \begin{equation}
   \epsilon(\vec{k})= \epsilon(0)+\frac{1}{2}\sum_{i=1}^3\left(\frac{\partial^2 \epsilon}{\partial{k_i}^2}\right)|_{k=0}{k_i}^2+...
   \label{eq:TaylorEnergie}
 \end{equation}
 Mit
 \begin{equation}
   \epsilon = \frac{\hbar^2 k^2}{2m}
   \label{eq:Epsilonausdruck}
 \end{equation}
 lässt sich Gl. \ref{eq:TaylorEnergie} zu einer Ellipsoidgleichung umstellen,
  welche die Flächen gleicher Energien im $\vec{k}-$Raum beschreibt.\\
  Für hohe Symmetrien erhält man kugelförmige Energieflächen
  \begin{equation}
    \epsilon(\vec{k})= \epsilon(0)+\frac{\hbar^2 k^2}{2m^{*}}.
   \label{eq:KugelEnergie}
  \end{equation}
 Das $m^{*}$ beschreibt hierbei die effektive Masse eines Kristallelektrons mit
 \begin{equation}
   m^{*}:=\frac{\hbar^2}{\left(\frac{\partial^2 \epsilon}{\partial{k_i}^2}\right)|_{k=0}} .
 \end{equation}
 Diese Definition hat den Vorteil, dass die Elektronen eines Kristall mit hoher Symmetrie durch die Bewegungsgleichung
 freier Teilchen beschrieben werden können. In einem periodischen Kristallpotential $V(\vec{r}+\vec{g})$ gilt
  also durch das Einführen der effektiven Masse für den Hamilton-Operator
 \begin{equation}
   \mathcal{H}=\frac{\hbar^2}{2m*}\Delta
   \label{eq:freiHamilton}
 \end{equation}
 \subsection{Rotation der Polarisationsebene}
  Das Phänomen der Drehung der Polarisationsebene eines linear
  polarisierten Lichtstrahls, welcher durch einen Kristall aus optisch aktivem Material fällt, wird zirkuläre Doppelbrechnung genannt.
  Dies kann auch bei inaktiver Materie erreicht werden, wenn ein äußers Magnetfeld anliegt.\\
  Dies kann durch die Annahme unterschiedlicher Phasengeschwindigkeiten für rechts- und
   linkspolarisiertes Licht im Kristall erklärt werden.\\
  Wird eine elektromagnetische Welle $E(z)$ in zwei unterschiedlich polarsierte Wellen zerlegt, die sich beide in $z$-Richtung
  ausbreiten aber unterschiedliche Wellenzahlen besitzen, kann die Welle durch
  \begin{equation}
    E(z)=\frac{1}{2}\left(E_{\text{R}}(z)+E_\text{{L}}(z)\right), \hspace{2cm}\text{mit}\. k_{\text{R}} \neq  k_{\text{L}}
    \label{eq:Wellezerlegt}
  \end{equation}
  beschrieben werden.\\
  Damit lässt sich der Winkel, um den die Polarisationsebene des Lichtstrahls rotiert wurde,
  nachdem er den Kristall der Länge $L$ durchquert hat, mit
  \begin{align}
    \theta &=\frac{L}{2} \left(k_{\text{R}}-k_{\text{L}}\right)\\
          &=\frac{L\omega}{2}\left(\frac{1}{v_{\text{ph}_{\text{R}}}}-\frac{1}{v_{\text{ph}_{\text{L}}}}\right)\\
          &=\frac{L\omega}{2\text{c}}\left(n_{\text{R}}-n_{\text{L}}\right)
  \end{align}
  beschreiben, wobei $v_{\text{ph}}=\frac{\omega}{k}$ die Phasengeschwindigkeit $n=\frac{c}{v_{\text{ph}}}$ der Brechungsindex ist.\\
  Die Doppelbrechung entsteht durch elektrische Dipolmomente, welche pro Volumen eine Polarisation
  \begin{equation}
    \vec{P}=\epsilon_0 \chi \vec{E}
  \end{equation}
  erzeugt.\\
  Die elektrische Suszeptibilität $\chi$ wird hierbei durch einen Tensor beschrieben.
  Dieser hat für doppeltberechende Materie die Form
  \begin{equation}
    \left( \chi \right)=
    \begin{pmatrix}
      \chi_{\text{xx}} & i\chi_{\text{xy}} & 0 \\
      -i \chi_{\text{xy}}& \chi_{\text{xx}} & 0 \\
      0& 0 & \chi_{\text{zz}}
      \end{pmatrix} .
    \end{equation}
  Unter Verwendung der Wellengleichung lässt sich zeigen, dass der Drehwinkel durch
  \begin{equation}
    \theta=\frac{L\omega}{2\text{c}n}\chi_{\text{xy}}
  \end{equation}
  gegeben ist.\\
  Um die Komponente $\chi_{\text{xy}}$ zu bestimmen, wird die Bewegungsgleichung
  eines gebundendes Elektron betrachtet.
  \begin{equation}
    m\frac{\text{d}^2 \vec{r}}{\text{dt}^2}+K\vec{r}= -\text{e}_0
    \vec{E}(r)-\text{e}_0\frac{\text{d}\vec{r}}{\text{dt}}\times \vec{B}
    \end{equation}
  Hierbei werden Dämpfungseffekte vernachlässigt und wegen des hohem $\omega$ wird nur eine
  Verschiebungspolarisation $\vec{P} \propto \vec{r}$ betrachtet.\\
  Damit lässt sich zeigen, dass
  die beiden nichtdiagonalelemente komplex konjungiert sind und sich $\theta$ in Abhängigkeit der
  Wellenlänge $\lambda$ durch
  \begin{equation}
    \theta(\lambda)=\frac{2\pi^2 \text{e}_0^3 \text{c}}{\epsilon_0}\frac{1}{m^2 \lambda^2 \omega{_0}^4}\frac{N B L}{n}
    \label{eq:WinkelLambda}
  \end{equation}
      beschreiben lässt.
  Für den Fall von freien Ladungsträgern wird $\omega_0 \rightarrow 0$ betrachtet und es ergibt sich für den Drehwinkel
\begin{equation}
  \theta_{\text{frei}}=\frac{\text{e}_0^3}{8 \pi^2 \epsilon_0 \text{c}^3}\frac{\lambda^2}{m^2}\frac{N L B}{n}
  \label{eq:Winkelfrei}
\end{equation}
