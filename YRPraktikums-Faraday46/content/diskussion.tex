\section{Diskussion}
\label{sec:Diskussion}
\subsection{Relativer Fehler}
Alle relativen Fehler wurden nach der Formel
\begin{equation*}
  \tilde{x} = \frac{ \lvert x_{lit} - x_{mess} \rvert}{\lvert x_{lit} \rvert}
  \cdot 100 \%
\end{equation*}
berechnet, dabei bezeichnet $x_{lit}$ den Literaturwert der Messgröße $x_{mess}$.

\subsection{Diskussion der vom Versuchsaufbau fehlerbehafteten Größen}
\label{sec:Fehler}
Der Verwendete Versuchsaufbau weist einige Fehler auf die besonders die gemessenen Winkel 
beeinflussen. Der verwendete Selektivverstärker kann nicht auf den Lichtzerhacker eingestellt 
werden, sonder nur der Zerhacker auf den Selektivverstärker. Dies hat zur Folge, dass die Signale 
die gemessen werden nicht so scharf gemessen werden können. Zudem sind die Signale der 
Photodetektoren nicht in Phase. Folglich ist auf dem Oszillokop keine Nullline zur erkennen. Es kann 
wenn überhaupt ein Minimum abgeschätz werden, da auch noch das Grundrauschen durch den nicht 
eingeregelten Selektivverstärker dazu kommt. Dazu kann keine Wechselfeldmethode verwendet werden, da 
der Magnet nicht umpolbar ist, dies ist nicht problematisch, könnte aber zu genaueren Ergebnissen 
führen. Zudem ist das Linsensystem nicht ausreichend zu justieren, so dass Intensität verloren geht 
und je nach Probe und Wellenlänge unterschiedlich starke Signale ermöglicht. 
Es lässt sich festhalten, dass die ermittelten Winkel mehr eine Abschatzung sind und einem vom 
Versuchsaufbau gegebenen systematischen Fehler unterliegen, der sich stark auf das Auflösevermögen 
der kleinen Winkelunterschiede auswirkt.

\subsection{Zur Faraday-Rotation an n-dotiertem und hochreinem GaAs}
Verwunderlich ist, dass das hochreine GaAs im Vergleich zum $N= \SI{1.2e18}{\per\centi\meter}$ 
n-dotiertem GaAs eine größere Farady-Rotation zeigt, da diese 
proportional zur Zahl der freien Ladungsträger ist. Alle anderen Faktoren sind durch den 
Versuchsaufbau gleich geblieben oder wurden bei dem Vergleich mitbetrachtet. Es lässt sich eine 
Verwächselung der Daten unsererseits ausschließen. Deshalb liegt es nah, dass es vom Versuchsaufbau 
her nicht möglich war beide Proben hinreichend genau zu untersuchen. Zudem war es besonders schwer 
bei der zuvor erwähnten Probe die Winkelunterschiede zu erkennen, dies wurde aber schon ausreichend in Kaptiel \ref{sec:Fehler} diskutiert.

\subsection{Zur Bestimmung der effektiven Masse} 
Um das Ergebnis mit dem Literaturwert $ m_{lit} ^* = 0.067 \cdot m_e$ aus Quelle \cite{effm} zu 
vergleichen, wird zu erst der Mittelwert der zuvor bestimmten effektiven Massen, unter Zurhilfenahme 
von \cite{numpy} nach den Methoden von \cite{Tipler}, bestimmt. Es wurde verwedet:
\begin{equation*}
\bar{x} = \frac{1}{n} \sum_i x_i \qquad \text{und} \qquad 
\Delta x = \sqrt{\frac{1}{n\left(n-1\right)} \sum_i ^n \left( x_i-\left<x\right>\right)^2 } \qquad . 
\end{equation*}
Daraus folgt die Mittlere effektive Masse $\bar{m^*} = \SI{1.1(15)e-6} \cdot m_e$. Daraus folgt ein 
relativer Fehler von \SI{99.998}{\percent}. So eine große Abweichung bestetigt die Annahme aus 
Kapitel \ref{sec:Fehler}, dass hier nur eine grobe Abschätzung der Winkelunterschiede vorliegt. 
In Kapitel \ref{sec:Auswertung} wurde dann zum Vergleich der längenomrierte Winkel betrachtet, 
welcher durch die geringe Proben dicke sehr groß ausfällt. Folglich fallen dann auch bei kleinen 
Winkelunterschieden große Werte an, somit scheint  auch eine große Abweichung von $ 10^{-3}$ 
Größenordnungen nicht unwahrscheinlich bei den Gegebenheiten die in Kapitel \ref{sec:Fehler} 
geschildert wurden. 
